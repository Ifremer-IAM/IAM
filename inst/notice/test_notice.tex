\documentclass[12pt, colorinlistoftodos]{article} 

\title{IAM:  Impact Assessment \\ Structure et mode opératoire du modèle}
\author{Maxime Jaunatre}

\usepackage{ifthen} 
\newboolean{draft}\setboolean{draft}{true} % display notes, stamp

\providecommand{\main}{.}  % *Modification: define file location
% \usepackage{\main/.tex/setup}

\usepackage{tabularx}
\usepackage{eurosym}

% % Beginning of the setup file !!!!
% set up file for new commands and main packages :
% \providecommand{\main}{.}  % *Modification: define file location
\usepackage[utf8]{inputenc} % encoding
\usepackage[T1]{fontenc}
\usepackage{geometry}
\geometry{a4paper} % format de feuille
\geometry{top=2.5cm, bottom=2.5cm, left=2cm, right=2cm} %marges
%\linespread{1.5} % interligne

\usepackage[table,dvipsnames*,svgnames]{xcolor} % allow to put colour in table and correct a bug in names
% option 'gray' emulate a gray scale print of the document
\usepackage{fancyhdr}
\pagestyle{fancy}
\chead{\csname @title\endcsname}
\lhead{}
\rhead{\includegraphics[width=2cm]{\main/.tex/Logo_ifr_y.jpg}}
\lfoot{page \thepage}\cfoot{}
\rfoot{\csname @author\endcsname ~-~\the\year}

\fancypagestyle{plain}{%
    \fancyhf{}% clear all header and footer fields
    %\fancyfoot[L, R]{page \thepage}{\csname @author\endcsname ~-~\the\year} 
    \lfoot{page \thepage}\cfoot{}
    \rfoot{\csname @author\endcsname ~-~\the\year}
    \renewcommand{\headrulewidth}{0pt}%\
    %\renewcommand{\footrulewidth}{0pt}
}

\ifthenelse{\boolean{draft}}{
    %\usepackage{pagecolor} \pagecolor{darkgray} \color{lightgray}
    \usepackage[color={[rgb]{0.96, 0.77, 0.19}}]{draftwatermark}
    \SetWatermarkScale{4}
    \SetWatermarkText{DRAFT}
    \usepackage[backgroundcolor = orange]{todonotes}
}{ \usepackage[disable]{todonotes} }

% using math
\usepackage{amstext}

\usepackage[hidelinks]{hyperref} % lien cliquables
\hypersetup{
colorlinks = true,
linkcolor = black,
citecolor = black,
urlcolor = blue
} % https://tex.stackexchange.com/questions/50747/options-for-appearance-of-links-in-hyperref
\usepackage{url}


\usepackage{array}
\usepackage{longtable}
\usepackage{float} 

\def\correction#1{%
    \abovedisplayshortskip=#1\baselineskip\relax\belowdisplayshortskip=#1\baselineskip\relax%
    \abovedisplayskip=#1\baselineskip\relax\belowdisplayskip=#1\baselineskip\relax}

\arrayrulewidth=1pt\relax
\tabcolsep=5pt\relax
\fboxsep=\tabcolsep\relax
\fboxrule=\arrayrulewidth\relax

\newcolumntype{A}[2]{%
    >{\minipage{\dimexpr#1\linewidth-2\tabcolsep-#2\arrayrulewidth\relax}\vspace\tabcolsep}%
    c<{\vspace\tabcolsep\endminipage}}

\newenvironment{Table}[4]{%
    \longtable{%
        |>{\centering$\displaystyle}A{#1}{1}<{$}% for inline equation
        |>{\centering}A{#2}{1.5}% for text
        |>{\centering}A{#3}{1.5}% for text
        |>{\centering$\displaystyle}A{#4}{1}<{$}% for inline equation
        |}\hline\ignorespaces}{%
    \endlongtable\ignorespacesafterend}

\newenvironment{nTable}[6]{%
    \longtable{%
        |>{ \addtocounter{rowcntr}{-1} \refstepcounter{rowcntr} \therowcntr }A{#1}{1.5}<{\addtocounter{rowcntr}{1}} 
        |>{\centering$\displaystyle}A{#2}{1}<{$}% for inline equation
        |>{\centering}A{#3}{1.5}% for text
        |>{\centering}A{#4}{1.5}% for text
        |>{\small \centering}A{#5}{1.5}% for text
        |>{\centering$\displaystyle}A{#6}{1}<{$}% for inline equation
        |}\hline\ignorespaces}{%
    \endlongtable\ignorespacesafterend}

\usepackage{ifthen}

\newcounter{rowcntr}[table]
\renewcommand{\therowcntr}{%
    \ifnum\value{rowcntr} > 0
    \ifnum\thetable = 1
        i\arabic{rowcntr}
    \else 
    \ifnum\thetable = 2
        p\arabic{rowcntr}
    \else
    \ifnum\thetable = 3
        t\arabic{rowcntr}
    \else
        \thetable. \arabic{rowcntr}
    \fi
    \fi
    \fi 
    \fi
}


\newcommand{\efmi}{%
    _{e,f,m,i}}
\newcommand{\efmc}{%
    _{e,f,m,c}}
\newcommand{\efm}{%
    _{e,f,m}}
\newcommand{\fm}{%
    _{f,m}}

\newcommand{\tabnl}{
    \tabularnewline\hline
}

\newcommand{\pref}[1]{(\ref{#1})}

% % EOF setup.sty

\newcolumntype{L}{>{\(\displaystyle }l<{\)}} % math-mode version of "l" column type

\begin{document}

\thispagestyle{plain}

\begin{figure}
    \includegraphics[width=\textwidth]{\main/.tex/header_ifre.png}
    \par ~ \par
    \begin{minipage}{\textwidth}
        \begin{center}
            {\huge \csname @title\endcsname }
        \end{center}
        \rule{7em}{.4pt}\par
        \csname @author\endcsname ~| UMR AMURE \hfill %\par 
        \href{mailto:maxime.jaunatre@ifremer.fr}{Mail} | \today
    \end{minipage}
\end{figure}
\hrule
%\ifthenelse{\boolean{draft}}{\listoftodos \hrule}{} %is this list present or not

\iffalse

\section{Introduction}

IAM (Impact Assessment Model) est un modèle bio-économique de simulation de dynamiques de pêcheries, 
intégrant des outils spécifiques d'aide à la décision dans le cadre de mises en application théoriques de mesures de gestion.  
Ce document constitue un support d'utilisation du modèle, décrivant les étapes de paramétrage et de lancement des simulations, 
exposant les méthodologies et équations fonctionnelles utilisées en arrière-plan, et analysant les possibilités offertes par l'outil. 
Une première partie  principalement théorique décrira l'architecture modulaire du modèle, 
détaillera les paramètres mis en jeu ainsi que les liens fonctionnels les unissant. Les différents outils de simulation 
greffés au modèle "basique" et les méthodologies associées seront également présentés dans cette partie. 
Plus techniquement, la méthode de constitution du fichier de paramétrage fera l'objet d'une seconde partie, 
alors qu'une troisième et dernière partie décrira la mise en application des simulations au sein d'un environnement R.



\section{Notations et structure schématique du modèle}

\subsection{Notations utilisées}

IAM est un modèle bio-économique à temps discret, multi-flottille, multi-métier, et multi-spécifique à composantes "âge"
 pour la partie biologique, et à composantes "catégorie commerciale" pour la partie économique. 
 Les paramètres du modèle peuvent ainsi se décliner selon 7 indices de définition distincts qui sont 
 décrits dans le tableau ci-dessous :

\begin{table}[h]
\centering
\begin{tabular}{|L| c |}
\hline
\textbf{Indice} & \textbf{Description} \\
 \hline
t & Indice temporel \\
f & Indice flottille \\
m_{bio} & Indice métier (paramètres biologiques) \\
m_{eco} & Indice métier (paramètres économiques) \\
e & Indice espèce \\
ie & Indice âge (dépend de l’espèce) \\
ce & Indice catégorie (dépend de l’espèce) \\
\hline
\end{tabular}
\label{neat}
\caption{Indices de déclinaison des variables numériques}
\end{table}

En pratique, ces indices vont également déterminer la structuration des objets R incarnant les inputs du modèle 
(voir partie 3\todo{ref to section}). Ces objets se présenteront en effet sous la forme de matrices multidimensionnelles formatées, 
incluant toute l'information nécessaire au modèle pour à la fois procéder aux calculs des indicateurs de sortie, 
mais aussi pour assurer la robustesse des implémentations. De surcroît, cette mise en forme spécifique, 
commune aux données d'entrée et de sortie, facilitera les traitements numériques ultérieurs qui pourront 
leur être appliqués. Notons que l'on peut distinguer deux types d'indices "métier", puisque le modèle 
prend en considération la possibilité de définir les paramètres biologiques et les paramètres économiques 
selon deux niveaux "métier" distincts. Dans ce cas, une matrice de correspondance entre ces deux niveaux de 
définition sera requise (matrice de dimension $m_{bio} \cdot m_{eco}$, voir partie 2\todo{ref to section}). Précisons également que plusieurs 
types de dynamiques de population pourront être appliquées à un stock donné ; elles sont pour le moment 
au nombre de 3, et sont les suivantes :

\begin{itemize}
    \item dynamique de population de type XSA (par défaut, annuelle et définie aux âges. Ex : Sole 8ab)
    \item dynamique de population de type SS3 (trimestrielle et définie par âge et cohorte. Ex : Merlu 8ab)
    \item dynamique de population de type SPiCT (modèle Pella-Tomlinson  annuel)
\end{itemize}

On peut rajouter à ces trois options la possibilité de ne pas considérer de dynamique de population. 
Les productions seront alors calculées sur  la base de débarquements par unité d'effort, considérés 
ou non comme constants au cours du temps. La mise en œuvre de chacune de ces dynamiques nécessitera un 
ensemble de paramètres spécifiques. 

\subsection{Schéma structurel du modèle bio-économique}

Le schéma de la page suivante décrit de manière synthétique et simplifiée le mode de fonctionnement du modèle bio-économique.
Il met à la fois en évidence la structure modulaire du modèle (chaque module étant représenté par un rectangle bleu), 
et les interactions entre ces modules au travers des relations existant entre les  paramètres mis en jeu. 
Il permet aussi de distinguer les modules "utilisateur" (rectangles bleus avec intitulés en couleur), 
qui serviront à orchestrer l'exécution des modules de simulation tout en permettant à l'utilisateur 
d'intervenir au cours du processus de simulation, des modules "passifs"  (rectangle bleus avec intitulés en noir) 
qui seront tributaires de l'action des précédents et qui composeront l'ossature du modèle. 
La dynamique de population ici considérée et illustrée est le modèle XSA. 
Pour plus de précisions concernant les interactions mises en jeu au sein des autres modèles de population, 
le lecteur pourra se référer aux équations détaillées de la partie 1.2\todo{ref to section}.

L'objectif des  chapitres qui vont suivre est de proposer une description exhaustive et organisée des 
processus mis en action durant l'étape de simulation. Il s'agira donc de fournir à la fois une synthèse 
des calculs effectués, mais aussi une vue des méthodologies utilisées et des différents outils mis à 
la disposition de l'utilisateur.


\section{Description et déroulement des modules}

La partie qui suit va s'attarder plus longuement sur les modules constituant le modèle bio-économique. 
Dans ce contexte, les paramètres d'entrée du module seront décrits (ils pourront tout aussi bien être 
des paramètres d'entrée du modèle que des paramètres de sortie d'autres modules), les sorties le seront également, 
et les équations décrivant le cheminement des calculs seront présentées. 

Notons que les paramètres initiaux et les variables calculées sont décrits ici à leur niveau maximal de précision. 
Il est bien évidemment concevable que certaines données ne soient pas disponibles à ce niveau de définition : 
l’ajustement nécessaire est généralement pris en compte dans l’implémentation (tests sur la compatibilité des 
dimensions entre variables, et correction du format si besoin est).

Précisons pour finir que  la dimension temporelle le plus souvent implicite dans les formulations suivantes 
peut être non seulement intégrée au stade initial de déclaration des paramètres d’entrées, mais également 
greffée à toute variable du modèle par le biais du module "scénario" permettant l'intervention de l'utilisateur 
à n'importe quel instant $t$.

\todo[inline]{add the great figure with cycle}

\subsection{Module "Mortalité par pêche et Survie des rejets"}

Ce module fait appel à deux sous-modules. Le premier procède à la ventilation de la donnée "mortalité" initiale 
disponible par une variable auxiliaire supposée corrélée afin d’affiner le niveau de définition accessible. 
Ce processus s’effectue le plus souvent au moyen d’une variable de type "captures", ou à défaut, 
d’une donnée "débarquements". Le deuxième sous-module va estimer initialement une capturabilité 
associée afin de l’appliquer au cours du processus de modélisation à une variable de contrôle "effort". 

\subsubsection{Sous-module "Allocation de la mortalité"}

Avant tout, précisons que ce qui va se rapporter à cette étape d'allocation ne concerne que les dynamiques \textbf{XSA} et \textbf{SPiCT}. 
La mortalité par pêche pour la dynamique SS3 est d'entrée renseignée déjà ventilée au niveau flottille-métier-âge-cohorte-saison 
requis, aucun travail d'allocation n'est donc nécessaire. Le modèle statique quant à lui s'affranchit de cet indicateur
 puisqu'il se base sur des données de LPUE déjà disponibles au niveau flottille-métier. 

L’objectif de ce module est d’aboutir à un niveau de définition maximal du paramètre de mortalité, à savoir idéalement 
une valeur par espèce, âge, flottille et métier. Il existe plusieurs manières d’obtenir ce niveau maximal  
(voir tableau 1.2.1.1.1\todo{ref} décrivant les procédés de ventilation implémentés). En partant d’une donnée "Mortalité" plus précise, 
on peut se contenter d’une variable de ventilation plus grossière. On peut également procéder en deux étapes successives 
avec deux variables distinctes si la variable initiale ne décline que la dimension "âge", et qu'il reste donc à 
intégrer les dimensions "métier", puis "flottille". Toutes ces possibilités sont offertes par le module implémenté, 
et celui-ci considérera au mieux les ventilations à opérer le cas échéant.

Si nécessaire, une première ventilation sera effectuée au moyen des variables Cmi et Ci préalablement renseignées dans 
les onglets "Espèce" du fichier de paramétrage (voir partie 2\todo{ref to section}) afin d'intégrer au minimum la dimension "métier" à 
la donnée de mortalité par pêche. On rappelle qu'il s'agira idéalement de données de captures en nombre, mais qu'une 
autre donnée pourra à défaut être utilisée. Si à ce stade, la mortalité possède le niveau complet requis, la procédure s'arrêtera. 
Sinon, la seconde ventilation s’opérera avec les variables $Ymi$ et $Lref_{fe}$ (variable présente dans les onglets de 
paramétrage économique et appliquée ici à la matrice "fm" d'allocation flottille-métier) afin d'inclure la dimension 
"flottille". Il revient à l'utilisateur de s'assurer de la pertinence des données assignées à chacune de 
ces variables afin que le processus de ventilation aboutisse à des résultats justes et optimaux 
(cf tableau 1.2.1.1.1 de la page suivante\todo{ref to section}).   

Note : la ventilation ne s'effectue plus en tant qu'étape préliminaire aux simulations lors du lancement du modèle, 
mais désormais  lors de la phase d'intégration des indicateurs visant à créer l'objet R de paramétrage. Le paramètre calculé 
$F_{fmi}$ apparaît donc maintenant parmi les paramètres "sources" au sein de l'objet, et est au même titre qu'eux 
directement utilisé par le modèle.  

$\bullet$ \textbf{Paramètres initiaux} (la description des paramètres d'entrée et l'équation de calcul associée dépeignant 
ici la procédure d'allocation ne doivent être considérées que comme un cas particulier d'une ventilation en une étape) 
\fi 

\iffalse
\begin{table}[h]
\begin{center}

\begin{flushleft}
    $\bullet$ XSA, SPiCT
    \end{flushleft}
    \begin{tabularx}{\textwidth}{|L| X | X|}
    \hline
    \textbf{Notation} & \textbf{Descritption} & \textbf{Source} \\
     \hline
    F_{e,i} & Coefficient de mortalité par pêche initial (ici, par espèce et par âge) & 
    Stock Assessment ($i=\{all\}$ pour modèle global SpiCT, et $\{i\}=\{0,1,\ldots\}$ pour modèle aux âges XSA)\\ \hline
    C_{e,f,m,i} & Variable de ventilation à un niveau de définition requis (ici, captures par espèce, flottille, métier et âge) & 
    SACROIS\\ \hline
    Ctot_{e,i} & Variable de ventilation totale (doit être définie sur l’intersection des niveaux des deux variables précédentes) & 
    SACROIS\\
    \hline
    \end{tabularx}




    \begin{flushleft}
        $\bullet$ SS3
    \end{flushleft}
    \begin{tabularx}{\textwidth}{|L| X | X|}
        \hline
        \textbf{Notation} & \textbf{Descritption} & \textbf{Source} \\
         \hline
        iniFq_{e,i} & Coefficient de mortalité par pêche total initial (par espèce, âge, saison et cohorte) & Stock Assessment \\ \hline
        Fq_{e,i} & Coefficient de mortalité par pêche total à t=1 (par espèce, âge, saison et cohorte) & Stock Assessment \\ \hline
        iniFq_{e,i} & Coefficient de mortalité par pêche total initial (par espèce, flottille, métier, âge, saison et cohorte) & Stock Assessment \\ \hline
        Fq_{e,f,m,i} & Coefficient de mortalité par pêche total à t=1 (par espèce, flottille, métier, âge, saison et cohorte) & Stock Assessment \\ \hline

        iniFqLwt_{e,i} & Coefficient de mortalité par pêche "poids débarqués" initial (par espèce, âge, saison et cohorte) & Stock Assessment \\ \hline
        FqLwt_{e,i} & Coefficient de mortalité par pêche "poids débarqués" à t=1 (par espèce, âge, saison et cohorte) & Stock Assessment \\ \hline
        iniFqLwt_{e,f,m,i} & Coefficient de mortalité par pêche "poids débarqués" initial (par espèce, flottille, métier, âge, saison et cohorte) & Stock Assessment \\ \hline
        FqLwt_{e,f,m,i} & Coefficient de mortalité par pêche "poids débarqués" à t=1 (par espèce, flottille, métier, âge, saison et cohorte) & Stock Assessment \\ \hline

        iniFqDwt_{e,i} & Coefficient de mortalité par pêche "poids rejetés" initial (par espèce, âge, saison et cohorte) & Stock Assessment \\ \hline
        FqDwt_{e,i} & Coefficient de mortalité par pêche "poids rejetés" à t=1 (par espèce, âge, saison et cohorte) & Stock Assessment \\ \hline
        iniFqDwt_{e,f,m,i} & Coefficient de mortalité par pêche "poids rejetés" initial (par espèce, flottille, métier, âge, saison et cohorte) & Stock Assessment \\ \hline
        FqDwt_{e,f,m,i} & Coefficient de mortalité par pêche "poids rejetés" à t=1 (par espèce, flottille, métier, âge, saison et cohorte) & Stock Assessment \\ \hline
        \hline
        \end{tabularx}
\end{center}
    \label{neat}
    \caption{Paramètres initiaux pour le sous-module "allocation de la mortalité par pêche"}
\end{table}

\begin{table}[h]
        \centering
    
        \begin{flushleft}
            $\bullet$ Variables calculées (XSA et SPiCT)
        \end{flushleft}
        \begin{tabularx}{\textwidth}{|L| c | X |L|}
        \hline
        \textbf{Notation} & \textbf{Type} & \textbf{Descritption} & \textbf{Source} \\
         \hline
        F_{e,f,m,i} & sortie & Coefficient instantané de mortalité par pêche ventilé initial (ici, par espèce, flottille, métier et âge) & 
        F_{e,f,m,i} = \frac{F_{e,i} \cdot C_{e,f,m,i}}{Ctot_{e,i}}\\ \hline
        Foth_{e,i} & Sortie & Mortalité par pêche initiale "autres flottilles" par espèce et âge. Entrée des modules \textit{Captures} et \textit{Dynamiques de populations} & 
        Foth_{e,i} = F_{e,i} - \sum_{f,m} F_{e,f,m,i} \\ 
        \hline
        \end{tabularx}
        \label{neat}
        \caption{Paramètres calculés pour le sous-module "allocation de la mortalité par pêche"}
        \end{table}

\fi
\newpage


% \begin{table}[H]
%     \begin{nTable}{0.05}{0.15}{0.1}{0.3}{0.0001}{0.38}
%     \textbf{N} & {\normalsize\textbf{Notation}} & \textbf{Type}& \textbf{Description}& \textbf{Unité} & \textbf{Equation} \tabnl
%     \label{qefmi} &Q\efmi & Sortie & & Capturabilité initiale (par espèce, flottille, métier et âge) & 
%      Q\efmi = \frac{F\efmi}{effort1_{f,m} \cdot effort2_{f,m}} \tabnl
%     \label{qei} &Q_{e,f,m,i} & Sortie & & Capturabilité initiale (par espèce, flottille, métier et âge) & 
%      Q_{e,f,m,i} = \frac{F_{e,f,m,i}}{effort1_{f,m} \cdot effort2_{f,m}} \tabnl
%      &Q_{e,f,m,i} & Sortie & & Capturabilité initiale (par espèce, flottille, métier et âge) &
%      Q_{e,f,m,i} = \frac{F_{e,f,m,i}}{effort1_{f,m} \cdot effort2_{f,m}} \tabnl
%     \end{nTable}
%     \caption{Input variable (i)} \label{tab:Q1}
%     \end{table}
  

% \begin{table}[H]
    \begin{nTable}{0.05}{0.17}{0.09}{0.2}{0.1}{0.4}
\textbf{N} & {\normalsize\textbf{Notation}} & \textbf{Type}& \textbf{Description} & \textbf{Unité} & \textbf{Equation} \tabnl
\label{Lefmc} & L_{e,f,m,c} & EC \tiny(bio) & Débarquements totaux par flottille, métier, espèce (et catégorie)
 & tonnes &  \tabnl
\label{Pefmc} & P_{e,f,m,c} & EC \tiny(marché) & Prix moyen par flottille, métier, espèce (et catégorie)
 & euro/kg &  \tabnl
\label{thetae} & theta_{e} & E & Multiplicateur (par espèce) à appliquer aux prix "débarquements" afin 
d’aboutir aux prix "rejets débarqués de taille commerciale". Inclus dans [0;1]. & - &  \tabnl
\label{LDefmc} & LD_{e,f,m,c} & EC \tiny(bio) & Rejets débarqués en poids, par flottille, métier, espèce modélisée et catégorie (taille et sous-taille)
 & tonnes &  \tabnl
\label{statLDorefm} & statLDor_{e,f,m} & EC \tiny(bio) & Rejets débarqués "autres" en poids, par flottille, métier et espèce statique
 & tonnes &  \tabnl
\label{statLDstefm} & statLDst_{e,f,m} & EC \tiny(bio) & Rejets débarqués "sous taille" en poids, par flottille, métier et espèce statique
 & tonnes &  \tabnl
\label{pste} & pst_{e} & E & Prix au kg de la partie "rejets débarqués sous taille" pour les espèces statiques
 & euro/kg &  \tabnl
\label{nbvf} & nbv_{f} & E & Nombre de navires par flottille & nombre &  \tabnl
\label{nbvfm} & nbv_{f,m} & E & Nombre de navires par flottille et métier & nombre &  \tabnl
\label{lcfm} & lc_{f,m} & E & Taxes de débarquements (en \% du CA) par flottille et métier & \% &  \tabnl
\label{lcdfm} & lcd_{f,m} &  & Taxes de débarquements des rejets "sous taille" (en \% du CA) par flottille et métier &  &  \tabnl
\label{tripLgthf} & tripLgth_{f} & E & Durée moyenne d'une marée par navire d'une flottille par an
 & heures &  \tabnl
\label{tripLgthfm} & tripLgth_{f,m} & E & Durée moyenne d'une marée par navire d'une flottille-métier & heures &  \tabnl
\label{nbTripf} & nbTrip_{f} & E & Nombre de marées annuel par navire d'une flottille & nombre &  \tabnl
\label{nbTripfm} & nbTrip_{f,m} & E & Nombre de marées annuel par navire d'une flottille-métier & nombre &  \tabnl
\label{nbdsf} & nbds_{f} & E & Nombre de jours de mer par navire d'une flottille par an & jours &  \tabnl
\label{nbdsfm} & nbds_{f,m} & E & Nombre moyen de jours de mer par navire d'une flottille-métier et par an & jours &  \tabnl
\label{effort1f} & effort1_{f} & E & Première composante d'effort par navire d'une flottille par an & - & nbTrip_{f\pref{nbTripf}} \text{ ou } nbds_{f\pref{nbdsf}} \tabnl
\label{effort1fm} & effort1_{f,m} & E & Première composante d'effort par navire d'une flottille-métier par an & - & nbTrip_{f,m\pref{nbTripfm}} \text{ ou } nbds_{f,m\pref{nbdsfm}} \tabnl
\label{effort2f} & effort2_{f} & E & Deuxième composante d'effort par navire d'une flottille par an & - & tripLgth_{f\pref{tripLgthf}} \text{ ou } 1 \tabnl
\label{effort2fm} & effort2_{f,m} & E & Deuxième composante d'effort par navire d'une flottille-métier par an & - & tripLgth_{f,m\pref{tripLgthfm}} \text{ ou } 1 \tabnl
\label{Lreffm} & Lref_{f,m} & E & Débarquements totaux de référence par flottille et métier & tonnes &  \tabnl
\label{cnbfm} & cnb_{f,m} & E & Effectif moyen par navire d'une flottille-métier & nombre &  \tabnl
\label{ovcfm} & ovc_{f,m} & E & Autres coûts variables DCF déduits des coûts de débarquements ($lc_{f,m\pref{lcfm}}$) par navire d'une flottille-métier & euro/an &  \tabnl
\label{fcfm} & fc_{f,m} & E & Coûts de carburant par navire d'une flottille-métier & euro/an &  \tabnl
\label{vffm} & vf_{f,m} & E & Prix du carburant par navire d'une flottille-métier & euro/L &  \tabnl
\label{cshrfm} & cshr_{f,m} & E & Part équipage (ratio du RAP) par navire d'une flottille-métier & \%RAP &  \tabnl
\label{cshrf} & cshr_{f} & E & Part équipage (ratio du RAP) par navire d'une flottille & \%RAP &  \tabnl
\label{cnbf} & cnb_{f} & E & Effectif moyen par navire d'une flottille & nombre &  \tabnl
\label{perscf} & persc_{f} & E & Coûts de personnel initiaux par navire d'une flottille & euro/an &  \tabnl
\label{eecf} & eec_{f} & E & Cotisations salariales totales par navire d'une flottille & euro/an &  \tabnl
\label{mwh} & mwh_{} & E & Salaire brut horaire  minimum national & euro/an &  \tabnl
\label{repf} & rep_{f} & E & Coûts entretien et réparation par navire d'une flottille & euro/an &  \tabnl
\label{gcf} & gc_{f} & E & Coût total engin par navire d'une flottille & euro/an &  \tabnl
\label{fixcf} & fixc_{f} & E & Autres coûts fixes par navire d'une flottille & euro/an &  \tabnl
\label{FTEf} & FTE_{f} & E & ETP par navire d'une flottille & hommes &  \tabnl
\label{depf} & dep_{f} & E & Amortissements par navire d'une flottille & euro/an &  \tabnl
\label{icf} & ic_{f} & E & Coût d'opportunité du capital par navire d'une flottille & euro/an &  \tabnl
\label{Kf} & K_{f} & E & Valeur d'assurance par navire d'une flottille & euro/an &  \tabnl
\label{invf} & inv_{f} & E & Coût d'investissement annuel par navire d'une flottille & euro/an &  \tabnl
\label{FTEfm} & FTE_{f,m} & E & ETP par navire d'une flottille-métier & hommes &  \tabnl
\label{perscCalc} & perscCalc_{} & E & Mode de calcul de la variable "coût du personnel" & - &
\textbf{0}\text{\footnotesize: salaires par marin fixés}\newline
\textbf{1}\text{\footnotesize: part équipage constante (cshr)}\newline
\textbf{2}\text{\footnotesize: part équipage constante calculée (ccwr)}\newline
\textbf{3}\text{\footnotesize: part équipage constante + salaire}\newline \text{\footnotesize marin supplémentaire fixé (cshr)}\newline
\textbf{4}\text{\footnotesize: part équipage constante calculée +}\newline \text{\footnotesize salaire marin supplémentaire fixé (ccwr)}
 \tabnl
\label{GVLreffm} & GVLref_{f,m} & E & CA moyen initial par navire d'une flottille-métier & euro/an &  \tabnl

\caption{Paramètres initiaux pour le module "Economique"}
    \end{nTable}

\newpage 

    \begin{nTable}{0.05}{0.17}{0.09}{0.2}{0.1}{0.4}
        \textbf{N} & {\normalsize\textbf{Notation}} & \textbf{Type}& \textbf{Description} & \textbf{Unité} & \textbf{Equation} \tabnl

        \label{ETinifm} & ETini\fm & I & Efficacité initiale de tri en tonnage par heure et homme, par flottille-métier &  tonnes/ \footnotesize h.homme &
        \frac{Lref\fm}{nbvIni\fm \cdot nbTripIni\fm} \times \frac{1}{tripLgthIni\fm \cdot cnbIni\fm}  \tabnl

        \label{uefm} & ue\fm & EC & Unité d'effort annuelle par navire d'une flottille-métier & - &
        effort1_{f,m\pref{effort1fm}} \cdot effort2_{f,m\pref{effort2fm}} \tabnl

        \label{uef} & ue_f & EC & Unité d'effort annuelle par navire d'une flottille & - &
        effort1_{f\pref{effort1f}} \cdot effort2_{f\pref{effort2fm}} \tabnl

        \label{fvoluefm} & fvolue\fm & EC & Volume de carburant par unité d'effort et par navire d'une flottille-métier & L/ue &
        \frac{ fc_{f,m\pref{fcfm}} }{ vf_{f,m\pref{vffm}} \cdot ue_{f,m\pref{uefm}} } \tabnl

        \label{ovcuefm} & ovcue\fm & EC & Autres coûts variables par unité d'effort et par navire d'une flottille-métier & \euro/ue &
        \frac{ ovc_{f,m\pref{ovcfm}} }{ ue_{f,m\pref{uefm}} } \tabnl

        \label{rtbsf} & rtbs_f & I & Reste à partager par navire d'une flottille & \euro/an &
        \sum_m ( GVLref_{f,m\pref{GVLreffm}} \cdot nbv_{f,m\pref{nbvfm}} \cdot (1 - lc_{f,m\pref{lcfm}}) 
        \cdot ovc_{f,m\pref{ovcfm}} \cdot fc_{f,m\pref{fcfm}} ) \times \frac{1}{ nbv_{f\pref{nbvf}} } \tabnl

        \label{ccwrf} & ccwr_f & EC & Part du coût de personnel en \% du RAP par flottille & \% &
        \frac{ persc_{f\pref{perscf}} }{ rtbs_{f\pref{rtbsf}} } \tabnl

        \label{operscf} & opersc_f & EC & Autres coûts de personnel par navire d'une flottille & \euro/an  &
        persc_{f\pref{perscf}} \cdot cshr_{f\pref{cshrf}} \cdot rtbs_{f\pref{rtbsf}} \tabnl
    \caption{Paramètres initiaux calculés pour le module "Economique"}
    \end{nTable}

\newpage

\begin{nTable}{0.05}{0.17}{0.09}{0.2}{0.1}{0.4}
    \textbf{N} & {\normalsize\textbf{Notation}} & \textbf{Type}& \textbf{Description} & \textbf{Unité} & \textbf{Equation} \tabnl

    & & & & &  GVLcom_{f,m,e} = \sum_{c\neq 999} P_{f,m,e,c(i2)} \times L_{f,m,e,c(i1)]} + \sum_{c\neq 999} \theta_{e(i3)} \times P_{f,m,e,c(i2]} \times LD_{f,m,e,c(i4)} \text{ si e modélisée} \newline
               GVLcom_{f,m,e} = P_{f,m,e(i2)} \times L_{f,m,e(i1)} + \theta_{e(i3)} \times P_{f,m,e(i2)} \times statLDor_{f,m,e(i5)} \text{ si e statique} \tabnl
    & & & & &  GVLst_{f,m,e} = \sum_{c \in 999} P_{f,m,e,c(i2)} \times L_{f,m,e,c(i1)]} + \sum_{c \in 999} \theta_{e(i3)} \times P_{f,m,e,c(i2]} \times LD_{f,m,e,c(i4)} \text{ si e modélisée} \newline 
               GVLst_{f,m,e} = pst_{e(i7)} \times statLDst_{f,m,e(i6)} \text{ si e statique} \tabnl
    & & & & &  GVLtot_{f,m,e} = GVLcom_{f,m,e(t1)} + GVLst_{f,m,e(t2)} \tabnl
    & & & & &  GVLtot_{f,m} = \sum_{e} GVLtot_{f,m,e(t3)} \tabnl
    & & & & &  GVLav_{f,m} = \frac{ GVLtot_{f,m(t4)} }{ nbv_{f,m(i9)} } \tabnl

    & & & & &  GVLtot_{f} = \sum_{m} GVLtot_{f,m(t4)} \tabnl
    & & & & &  GVLav_{f} = \frac{ GVLtot_{f(t6)} }{ nbv_{f(i8)} } \tabnl
    & & & & &  NGVLav_{f,m} = \frac{ \sum_{e} GVLcom_{f,m,e(t1)} \times \left( 1 - lc_{f, m(i10)} \right) }{ nbv_{f, m(i9)} }
        + \frac{ \sum_{e} GVLst_{f, m, e(t2)} \times \left( 1 - lcd_{f,m(i11)} \right) }{ nbv_{f, m(i9)} }
         \tabnl
    & & & & &  cnb_{f,m} = \frac{ \sum_{e,c} \left( L_{f,m,e,c(11)} + LD_{f,m,e,c(i4)} \right) }{ ETini_{f,m(p1)} \times nbv_{f, m(i9)} \times nbTrip_{f,m(i15)} \times tripLgth_{f,m(i13)} } \tabnl
    & & & & &  cnb_{f} = \frac{ \sum_{m} \left( cnb_{f,m(i9)} \times nbv_{f, m(i9)} \times nbTrip_{f,m(i15)} \times tripLgth_{f,m(i13)} \right) }{ nbv_{f(i8)} \times nbTrip_{f(i15)} \times tripLgth_{f(il2)} } \tabnl


    & & & & &  rtbs_{fm} = NGVLav_{f,m(t8}) - ( ovcue_{f,m(t4)} + fvolue_{f,m(p5)} \times vf_{f,m(i26)} ) \times ue_{f,m(p2)} \tabnl
    & & & & &  rtbs_{f} = \frac{ \sum_{m} \left( rtbs_{f,m(tl4)} \times nbv_{f,m(i9)} \right) }{ nbv_{f(i8)} }  \tabnl
    & & & & &  cshrT_{f,m} = cshr_{f,m(i27)} \times rtbs_{f,m(tll \text{ si perscCalc }=0 \text{ ou } 1 \text{, NA sinon)} } \tabnl
    & & & & &  cshrT_{f} = cshr_{f(i28)} \times \frac{ rtbsIni_{f(p6)} }{ cnbIni_{f(i29)} } \times cnb_{f(i10)} 
    \newline \text{si } perscCalc_{(i42)} = 0 \text{ (salaires par marin fixé)} \newline
               cshrT_{f} = cshr_{f(i28)} \times rtbs_{f(tl2)}
    \newline \text{si } perscCalc_{(i42)} = 1 \text{ (part équipage constante)} \newline
               cshrT_{f} = ccwr_{f(i28)} \times rtbs_{f(tl2)}
    \newline \text{si } perscCalc_{(i42)} = 2 \text{  (part équipage constante)} \newline
               cshrT_{f} = cshr_{f(i28)} \times ( rtbs_{f(tl2)} + \frac{ rtbsIni_{f(p6)} }{ cnbIni_{f(i29)} } \times ( cnb_{f(i10)} - cnbIni_{f(i29)} ) )
    \newline \text{si } perscCalc_{(i42)} = 3 \text{ (salaires marin suppl. fixé)} \newline
               cshrT_{f} = ccwr_{f(i28)} \times ( rtbs_{f(tl2)} + \frac{ rtbsIni_{f(p6)} }{ cnbIni_{f(i29)} } \times ( cnb_{f(i10)} - cnbIni_{f(i29)} ) )
    \newline \text{si } perscCalc_{(i42)} = 4 \text{ (salaires marin suppl. fixé)
                           } \tabnl
    & & & & &  ncshr_{f} = cshrT_{f(t14)} - eec_{f(i31)} \tabnl

    & & & & &  ocl_{f} = mwh_{(i32)} \times cnb_{f(tl0)} \times nbTrip_{f(i14)} \times tripLgth_{f(i12)} \tabnl
    & & & & &  cs_{f} = cshrT_{f(t14)} - ocl_{f(t16)} \tabnl
    & & & & &  csTot_{f} = cs_{f(t17)} \times nbv_{f(il8)} \tabnl
    & & & & &  gva_{f} = rtbs_{f(tl2)} - rep_{f(i33)} - gc_{f(i34)} - fixc_{f(i35)} \tabnl
    & & & & &  gvamargin_{f} = \frac{ gva_{f(t21)} }{ GVLav_{f(t7)} } \tabnl


    & & & & &  gvaFTE_{f} = \frac{ gva_{f(t21)} }{ FTE_{f(i36)} } \tabnl
    & & & & &  ccw_{f} = cshrT_{f(t14)} + opersc_{f(p8)} \newline \text{si } perscCalc_{(i42)} = 1,0 \text{ ou } 3 \newline 
               ccw_{f} = cshrT_{f(t14)} \newline \text{si } perscCalc_{(i42)} = 2 \text{ ou } 4 \tabnl
    & & & & &  ccwCr_{f} = \frac{ ccw_{f(t22)} }{ cnb_{f(t10)}}  \tabnl
    & & & & &  wageg_{f} = \frac{ cshrT_{f(t14)} }{ cnb_{f(t10)} } \tabnl
    & & & & &  wagen_{f} = \frac{ ncshr_{f(t15)} }{ cnb_{f(t10)} } \tabnl

    & & & & &  wagegFTE_{f} = \frac{ wageg_{f(t24)} }{ FTE_{f(i36)} } \tabnl
    & & & & &  wagegh_{f} = \frac{ wagegFTE_{f(t26)} }{ nbTrip_{f(i14upd)} \times tripLgth_{f(i12upd)} } \tabnl
    & & & & &  gp_{f} = gva_{f(t19)} - ccw_{f(t22)} \tabnl
    & & & & &  gpmargin_{f} = \frac{ gp_{f(t28)} }{ GVLav_{f(t7)} } \tabnl
    & & & & &  ncf_{f} = gp_{f(t28)} - dep_{f(i37)} \tabnl


    & & & & &  np_{f} = ncf_{f(t30)} - ic_{f(i38)} \tabnl
    & & & & &  npmargin_{f} = \frac{ np_{f(t31)} }{ GVLav_{f(t7)} } \tabnl
    & & & & &  prof_{f} = \newline \text{"Hight" : si } npmargin_{f(t32)} > 10\% \newline
                                   \text{"Reasonable" : si } npmargin_{f(t32)} \in [0\%;10\%] \newline
                                   \text{"Weak" : si } npmargin_{f(t32)} < 0\% \newline \tabnl
    & & & & &  npmarginTrend_{f,t} = \frac{ npmargin_{f(i32)} }{ \frac{1}{5} \sum_{T\in\{T-5, \ldots, t-1\}} npmargin_{f(i32)} } \newline 
                                    \text{"Improved" : si } devTrend > 6\% \newline
                                    \text{"Stable" : si } devTrend \in [-5\%;6\%] \newline
                                    \text{"Deterioration" : si } devTrend < -5\% \newline\tabnl
    & & & & &  ssTot_{f} = gp_{f(t28)} \times nbv_{f(i8)} \tabnl

    & & & & &  ps_{f} = nbv_{f(i8)} \times \left( cs_{f(t17)} + gp_{f(t28)} \right) \tabnl
    & & & & &  sts_{f} = \sum_{m} lc_{f,m(i10)} \times GVLav_{f,m(t5)} \times nbv_{f,m(i9)} \tabnl
    & & & & &  BER_{f} = \frac{ GVLtot_{f(t6)} \times ( fixc_{f(i35)} + ic_{f(i38)} + dep_{f(i37)} ) \times nbv_{f(i8)} }{ \sum_{m} (  NGVLav_{f,m(t8}) - ( fvolue_{f,m(p5)} \times vf_{f,m(i26)} + ovcue_{f,m(p4)} ) ) \times nbv_{f, m(i9)} }\tabnl
    & & & & &  CRBER_{f} = \frac{ GVLav_{f(t7)} }{ BER_{f(t38)} } \tabnl
    & & & & &  fuelEff_{f} = \frac{ \sum_{m} fvolue_{f,m(p5)} \times ue_{f,m(p2)} \times nbv_{f, m(i9)} }{ \sum_{m,e,c} ( L_{f,m,e(i1)} + LD_{f,m,e,c(i4)} ) + \sum_{m,e} ( L_{f,m,e(i1)} + statLDor_{f,m,e(i5)} + statLDst_{f,m,e(i6)}) }\tabnl


    & & & & & fvolGVA_{f} = \frac{ \sum_{m} fvolue_{f,m(p5)} \cdot ue_{f,m(p2)} }{ gva_{f(t19)} } \tabnl
    & & & & & gpGVA_{f} = \frac{ gp_{f(t28)} }{ gva_{f(t19)} } \tabnl
    & & & & & gvlK_{f} = \frac{ GVLav_{f(t7))} }{ K_{f(t39)} } \tabnl
    & & & & & gpK_{f} = \frac{ gp_{f(t28)} }{ K_{f(t39)} } \tabnl
    & & & & & RoFTA_{f} = \frac{ ncf_{f(t30)} }{ K_{f(t39)} } \tabnl

    & & & & & ROI_{f} = \frac{ gp_{f(t28)} - inv_{f(i40)} }{ inv_{f(i40)} } \tabnl
    & & & & & npK_{f} = \frac{ np_{f(t31)]} }{ K_{f(t39))} } \tabnl
    & & & & & gvlCNBue_{f} = \frac{ GVLav_{f(t7)} }{ cnb_{f(t10)} \times ue_{f(p3)} } \tabnl

    \caption{Paramètres initiaux pour le module "Economique"}
        \end{nTable}

\end{document}